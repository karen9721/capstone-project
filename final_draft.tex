\documentclass[12pt, a4paper,oneside]{book}
\newcommand{\re}{\mathrm{e}}
\newcommand{\ri}{\mathrm{i}}
\newcommand{\rd}{\mathrm{d}}
\def\semicolon{\nobreak\mskip2mu\mathpunct{}\nonscript\mkern-\thinmuskip{;}
\mskip6muplus1mu\relax} % This defines the semicolon command

        % allows index generation
\usepackage{graphicx}        % standard LaTeX graphics tool
                             % when including figure files
\usepackage{multicol}        % used for the two-column index




\usepackage{color,tikz}
%\usepackage[unicode,bookmarks,bookmarksopen,bookmarksopenlevel=2,colorlinks,linkcolor=blue,citecolor=green]{hyperref}

\usepackage{amsmath,eucal,amssymb}
\usepackage{mathrsfs,graphicx,texdraw}
\usepackage{fancyhdr,framed}
\usepackage{tikz, tikz-3dplot}
\usepackage{tkz-euclide}
\usetikzlibrary{decorations.fractals}
\usetikzlibrary{decorations.footprints}

\usepackage{palatino}

\usepackage[latin1]{inputenc}

\usepackage[T1]{fontenc}
%\usepackage[dvips]{graphicx}
%\usepackage{times}


\definecolor{mygreen}{HTML}{622567} %%% Purple
\definecolor{Gray}{HTML}{333333} %%% Gray

\definecolor{myred}{HTML}{D55C19} %%%EssexOrange
\definecolor{myblue}{HTML}{007A87} %%%Seagrass
\definecolor{Mint}{HTML}{35C4B5} %%% Mint


\def\grole{\mathrel{\mathchoice {\vcenter{\offinterlineskip\halign{\hfil
$\displaystyle##$\hfil\cr>\cr\noalign{\vskip-1.5pt}<\cr}}}
{\vcenter{\offinterlineskip\halign{\hfil$\textstyle##$\hfil\cr
>\cr\noalign{\vskip-1.5pt}<\cr}}}
{\vcenter{\offinterlineskip\halign{\hfil$\scriptstyle##$\hfil\cr
>\cr\noalign{\vskip-1pt}<\cr}}}
{\vcenter{\offinterlineskip\halign{\hfil$\scriptscriptstyle##$\hfil\cr
>\cr\noalign{\vskip-0.5pt}<\cr}}}}}

\newenvironment{rcases}
  {\left.\begin{aligned}}
  {\end{aligned}\right\rbrace}

\newenvironment{lcases}
  {\left\lbrace\begin{aligned}}
  {\end{aligned}\right.}

\newtheorem{theorem}{Theorem}[section]
\newtheorem{lemma}[theorem]{Lemma}
\newtheorem{proposition}[theorem]{Proposition}
\newtheorem{corollary}[theorem]{Corollary}
\newtheorem{definition}[theorem]{Definition}
\newtheorem{example}[theorem]{Example}
\newtheorem{remark}[theorem]{Remark}

\newenvironment{proof}[1][Proof]{\begin{trivlist}
\item[\hskip \labelsep {\bfseries #1}]}{\end{trivlist}}
\newenvironment{solution}[1][Solution]{\begin{trivlist}
\item[\hskip \labelsep {\bfseries #1}]}{\end{trivlist}}

\newcommand{\qed}{\nobreak \ifvmode \relax \else
      \ifdim\lastskip<1.5em \hskip-\lastskip
      \hskip1.5em plus0em minus0.5em \fi \nobreak
      \vrule height0.75em width0.5em depth0.25em\fi}

\numberwithin{equation}{section}
\def\Ad{{\mbox{Ad}}}
\def\im{{\mbox{Im}}}
\def\Re{{\mbox{Re}\;}}
\def\ad{\mathrm{ad\,}}
\def\openone{\leavevmode\hbox{\small1\kern-3.3pt\normalsize1}}
\def\Res{\mathop{\mbox{Res}\,}\limits}
\def\biglb{\big[\hspace*{-.7mm}\big[}
\def\bigrb{\big]\hspace*{-.7mm}\big]}


\def\bigrbt{\mathop{\bigrb }\limits_{\widetilde{\;}}}
\def\biggrbt{\mathop{\biggrb }\limits_{\widetilde{\;}}}
\def\Bigrbt{\mathop{\Bigrb }\limits_{\widetilde{\;}}}
\def\Biggrbt{\mathop{\Biggrb }\limits_{\widetilde{\;}}}

\def\bPhi{\mathbf{\Phi}}
\def\bM{\mathbf{M}}
\def\bm{\mathbf{m}}
\def\bbbc{{\Bbb C}}
\def\bbbr{{\Bbb R}}
\def\bbbz{{\Bbb Z}}
\def\bbbs{{\Bbb S}}
\def\diag{\mbox{diag}\,}
\def\tr{\mbox{tr}\,}

\textwidth=17cm   \textheight=24.5cm \voffset=-2cm

\evensidemargin=-0.5cm \oddsidemargin=-0.5cm

 \renewcommand{\baselinestretch}{1.5}

%%%%%%%%fancy header%%%%%%%%%%%%%%%%%

\usepackage{fancyhdr}
\pagestyle{fancy}
\usepackage{calc}
\newlength{\pageoffset}
\setlength{\pageoffset}{0cm}% use whatever you like
\fancyheadoffset[LE,RO]{\pageoffset}
\renewcommand{\chaptermark}[1]{\markboth{#1}{}}
\renewcommand{\sectionmark}[1]{\markright{\thesection\ #1}}
\fancyhf{}
\fancyhead[LE]{\makebox[\pageoffset][l]{\thepage}\hfill\leftmark}
\fancyhead[RO]{\rightmark\hfill\makebox[\pageoffset][r]{\thepage}}
\fancypagestyle{plain}{%
    \fancyhead{} % get rid of headers
    \renewcommand{\headrulewidth}{0pt} % and the line
}

%%%%%%%%%%%%%%%%%%%%%%%%%%%%%%%%%%%%

\arraycolsep=2pt

%%%%Fancy Chapter%%%

%\usepackage{lmodern}
\usepackage{graphicx}
\usepackage{xcolor}
\usepackage{titlesec}
\usepackage{microtype}
\usepackage{lipsum}

%\definecolor{myblue}{RGB}{0,82,155}

\titleformat{\chapter}[display]
  {\normalfont\bfseries\color{myred}}
  {\filleft\hspace*{-60pt}%
    \rotatebox[origin=c]{90}{%
      \normalfont\color{black}\Large%
        \textls[180]{\textsc{\chaptertitlename}}%
    }\hspace{10pt}%
    {\setlength\fboxsep{0pt}%
    \colorbox{myred}{\parbox[c][3cm][c]{2.5cm}{%
      \centering\color{white}\fontsize{80}{90}\selectfont\thechapter}%
    }}%
  }
  {10pt}
  {\titlerule[2.5pt]\vskip3pt\titlerule\vskip4pt\LARGE\sffamily}

% hyperref should be loaded after all other packages
% aliascnt is needed to get \autoref (from hyperref) to work correctly with custom amsthm theorems
\usepackage{aliascnt}
\usepackage[colorlinks,linkcolor=myblue,citecolor=mygreen]{hyperref}


%%% Make Necessary Changes Here Only %%%

\newcommand{\MyName}{Your Name} %% Change this to your name %%
\newcommand{\CapstoneVariant}{MAXXX} %% Change this to the capstone project variant you enrolled %%
\newcommand{\ProjectTitle}{Title of Your Project} %% Change this to your project title %%
\newcommand{\SuperviserName}{Your Supervisor}  %% Change this to your project supervisor %%

%%%%%%%%%%%%%%%%%%%%%%%%%%

%%% Do not change anything here %%%
\begin{document}

\thispagestyle{empty}

\begin{minipage}{0.2\textwidth}
\centerline{\includegraphics[width=.4\textwidth]{essex} }
\end{minipage}
\begin{minipage}{0.8\textwidth}

$ \qquad \qquad \qquad ${\LARGE \bf \sl University of Essex}

{\LARGE \bf Department of Mathematical Sciences}

\end{minipage}

\begin{center}

\noindent\textcolor{myred}{\rule{\linewidth}{4.8pt}}

\vspace{1cm}

{\LARGE \sc MA838: Capstone  Project Dissertation}

\vspace{1.5cm}

{\Huge{\color{myblue} Communicating Mathematics \\ and Data Science}}

\vspace{1.5cm}


{\Large \bf Karen Yim \\ 1705960}

\vspace{1.5cm}

%\centerline{\includegraphics[width=1\textwidth]{Figure}}
\vspace{3cm}

\vspace{1.5cm}

{\Large {Supervisor:} {\color{mygreen} \bf  Dr Joe Bailey}}

\vspace{.25cm}

\noindent\textcolor{myred}{\rule{\linewidth}{4.8pt}}

\vspace{1cm}
{\Large \today }\\[4pt]
{\Large Colchester}

\end{center}

\newpage

\section*{Abstract}
%Abstract of approximately 300 words summarizes the aims, scope, and the conclusions of a dissertation.

%Abstract (500 words roughly), brief rationale (meaning behind) why are you doing the study, method (explain audience, age, whatever, like experiment), results (data findings), discussions (conclusions, explain what the findings mean in context, how is it applied to the research project), limitations (what would you do to improve, or further research)

The aim of the project is to conduct a study on how to engage with students on communicating mathematics and data science. I've chosen this project as of previous interest in mathematics education and the possibility of showcasing new career options and topics to share to different age groups. The main project is communicating mathematics and data science which will be divided into sections as to use two different engagement methods. As a result

\tableofcontents
%\listoffigures
%\listoftables

%%% Do not change anything  above %%%
\chapter{Introduction}\label{ch:1}
%% You can start with some introduction of your project and background.
My project consists of two sections, the first is communicating mathematics to Year 9 students on the topics of Origami and Mathematics. The second is to communicate data science to Sixth Formers on the {\color{red}\textbf{topic of data science with the Spotify API.}} (need to reword) {\color{blue}\textbf{note: this might be better suited to another section such as abstract perhaps.}}

To communicate mathematics and data science is to ask - how did I find mathematics in secondary school and how was it interesting to learn? Many have spoken about the difficulties of understanding mathematics as it can be too "logical" that the more creative students were unable to comprehend. However, I believe this to be {\color{red}\textbf{wrong}} if the teacher was able to evidently show the similarities between the arts and sciences. As of this, I have chosen to put my dedication into mathematics education and specifically on combining and sharing to others the benefits and/of how similar arts and sciences are. In this case, it will be origami and mathematics to communicate mathematics and using the Spotify API to communicate data science.

{\color{blue}\textbf{Maybe put the bottom part of introduction into a new section known as Background?}}

Firstly, {\color{blue}\textbf{(perhaps remove "Firstly", not sure yet)}} prior to choosing the project, I was previously interested in learning new things to expand my horizon such as different maths topics. Hence, I spent a number of hours watching numerous YouTube videos in the background or sometimes more focused in the topic I was intrigued by. When the project briefs were shown, I had an interest in the project by Professor Abdel Salhi - "Error modelling in Origami construction". Consequently, I wanted to learn more about what origami maths was and proceeded to conducting a research on the topic.

Unfortunately, I did not choose the project in the end as I realised there was another project that could enable me to show the creative side of maths through the project "Communicating Mathematics/Data Science" by Dr. Joe Bailey. This latter project would subsequently let me explore the different engagement methods of communication as well as showcase my knowledge and interest of creativity bonding with mathematics. {\color{red}\textbf{(This latter project would subsequently hone my knowledge and connect with my musical side to mathematics.)}}

Prior to applying for university, I was captivated by music and maths, which the composer John Cage had influenced me to this joint degree. His compositions were inspired by one number, or a set of numbers which would result in his piece showcasing that specific number. My interest for this topic further expanded when I came across the book "Music of the Spheres".

This book showed that the arts and sciences were undeniably closely related and similar before they diverged into their respective areas in this modern society. Those who are from either of the fields may have discrepancies, however they would not be able to work without the other. This {\color{red}\textbf{meant}} (change the word) that to [make the findings of science and logic required creativity and vice versa. In this modern society, many may believe this to be preposterous, it would be outrageous that they are linked. This project aims to {\color{red}\textbf{affirm}} (change the word) that arts and sciences go hand-in-hand with each other.

\newline {\color{blue}\textbf{another version -------------------------------}}
\newline To write about the project is to first introduce what the project is. This project is on communicating mathematics and data science with a general overview and a focus on using two teaching methods to different audiences as a comparison between different forms of teaching styles. This dissertation will expand on the process of understanding effective and engaging pedagogy in mathematics, different types of teaching styles and methods, preparation and conclusions of chosen pedagogy styles, and the use of mathematical language and how to interact with different audiences.

To communicate mathematics and data science is to ask - how did I find mathematics in secondary school and how was it interesting to learn? Many have spoken about the difficulties of understanding mathematics as it can be too "logical" that the more creative students were unable to comprehend. However, I believe this to be {\color{red}wrong} if the teacher was able to evidently show the similarities between the arts and sciences. As of this, I have chosen to put my dedication into mathematics education and specifically on combining and sharing to others the benefits and/of how similar arts and sciences are. In this case, it will be origami and mathematics to communicate mathematics and using the Spotify API to communicate data science.

As of this, to first understand on how to communicate the two subjects, I will conduct a research on pedagogy for mathematics. Then, to communicate mathematics, I've chosen to create a presentation as to engage with Year 9 students (13-14 year olds) with handouts and interactive activities to learn about Pythagoras' Theorem using origami. Next, I will communicate data science in the form of a blog post detailing the specifics for a Spotify API (Application Programming Interface) aimed towards Sixth Formers (16-18 year olds). Finally, the last chapter entails the project findings, conclusions, and project reflections.

\chapter{Literature Review/Pedagogy Research}\label{ch:2}

Chapter 1 will detail the background research carried out in preparation for the project of communicating mathematics and data science. In order to teach mathematics, one must undertake preparation to fully understand {\color{red}\textbf{the aims of the project and how to proceed.}} (how to teach and what is required before teaching a class.)  As a result, the first preparation was a series of reviewing different academic journal, books, online media and personal experience on teaching maths.

\newline {\color{blue}\textbf{another version -------------------------------}}

When conducting the literature review, I started it off with what I had known about teaching in mathematics in different forms of media, however I had not known what pedagogy meant. Hence, the first step was to understand the word "Pedagogy".
The definition of pedagogy (from Oxford Language Dictionary) is "The method and practice of teaching, especially as an academic subject or theoretical concept.".

Prior to the project and researching pedagogy, I was always interested in learning new topics and expanding my knowledge of(or on?) different maths topics. As of this, I was exposed to various speakers who presented numerous topics on mathematics. Taken from previous {\color{red}\textbf{experiences}}, I was revisiting and recollecting my {\color{red}\textbf{thoughts and feelings}} when attending these talks - either watching pre-recorded videos, live videos online, or face-to-face settings. Albeit it comprises of many years of opinions, much of them combine as there had not been significant change for certain teaching styles and methods.

To expand on the experiences, it includes talks, masterclasses, webinars, seminars, summer schools and workshops, open day talks, induction talks, and live videos. The former few events are primarily classified as engagement activities albeit some were more informative, it was not necessarily as much a teaching event as university lectures or inaugural lectures. Hence, I would classify most of them as a majority to be an engagement-style {\color{red}\textbf{event}}. This had provided significant support to my project and for the rest of the time to conduct the research on other platforms for the project. Next, we will discuss the different age categories which can help to showcase the styles and methods comparisons in more detail.

\section{Teaching Styles}\label{sec:2.1}
To understand the different teaching styles, I had been researching primarily on the resource of YouTube as well as from personal experience. As a result, it can be detailed that there are lectures, lessons, engagement activities, masterclasses, talks, summer schools, and workshops. The level of complexity (both reading and speaking) correlates with the age group and education so that the learner can understand the concept. In this case, the earliest memory would be whilst at secondary school aged 11-16.

\subsection{Talks and Masterclasses}
Firstly, during secondary school, I was in the year groups of Year 9/10/11, or it might have been in the first year of sixth form at age 16/17, where I attended masterclasses at Greenwich University for mathematics masterclasses. This was interesting because we would have had some knowledge of A-level maths and was learning more concepts. This would be an engagement activity and had been explained with a depth that A-level students would understand but with still some difficulty in the concepts that are aimed towards higher education at university. Whilst it was difficult to understand everything, it was first explained in simpler terms and continued to be a fun and engaging topic enabling students to be focused and interactive during the masterclass. It involved doing activities in between explanations and was for the day with breaks and tours of the university campus.

There are a number of factors which can affect the quality and engagement level of students depending on the presenter and their ability to draw their audience. For example, as previously mentioned, their use of complex language can confuse and tire the audience. This could be using jargon in an public audience setting where few know what the words mean, but another instance could be showing the formula without explanation. Both illustrates a lack of understanding and adapting to their audience in order to capture their attention throughout the event. If the speaker researches beforehand and observes their audience, they can adapt and be flexible. As a result, a feedback loop could be created if students found their teacher engaging, persuading students to self-study and improve their understanding on the topic of interest.

In general, when delivering a talk, there is the common use of hand gestures and eye contact with the audience. Consequently, it keeps the audience engaged and focused on what the speaker is saying. Some speakers like to include humour to lighten the {\color{red}session} and ensure that it is not too heavy on the delivery of content. As for other speakers, due to time constraints and the amount of content to be covered, would tend to stay focused and leave questions until the end and would sometimes speak faster without realising. This is particularly evident for those who have pre-made presentations or have handwritten their notes before {\color{red}doing} the lecture. Due to this, it can be confusing and overwhelming for the audience as there is a lot of information to take in yet they do not have the time to take it in, absorb and to understand.

\subsection{Summer Schools and Workshops}
Next were talks from participating in summer schools which would be a duration of a week. Although these were not mathematical based, it provided insight on how to keep students engaged and {\color{red}absorbed} by the information. However, another event at university included the period of applying to universities in which Royal Holloway had an induction day for mathematics and music. On the induction day, there was an introduction to mathematics and what is included as well as a short lecture. Whilst it was already pre-assigned to have specific events during the day, it was an engagement activity but no longer aimed for younger aged students, therefore, it had a serious atmosphere similar to a lecture. It was informative, yet not boring as it was of interesting topics to stay engaging and understandable.

In 2019, I attended the Essex Mathematical Sciences Summer School, participating in the R-courses, Basic Data Science Skills for Industry, and a workshop: Predictive Data Science Impact in the Digital Industry. During this period, there was mathematical concepts involved as well as laboratory work. As of this, I was able to experience different teaching styles and methods. Generally, summer schools are more casual and

\subsection{Seminars and Webinars}
Similar to talks and masterclasses, seminars and webinars also cover {\color{red}a lot} of content and depending on what the topic is, seminars and webinars are more interactive in smaller classes and so it can be quite fun and engaging. For example, there were a few seminars delivered in the DMS (Department of Mathematical Sciences). These were delivered to a range of age groups from university undergraduate students, postgraduate students and?/to the staff members, depending on who were interested in joining. Majority of the time, there is intensive content and explanation, and with a time frame of just under an hour, it is hard to be able to explain in depth of their talk.
However, by using more technical terms, it can be understood by some lecturers and staff, but not all if it is not their main focus/specialty.

Due to this, for students, it is to try and learn new things. Whilst coming out of the lecture not fully understanding the entire mathematical content, it does increase the feedback loop of potentially being interested in the topic itself and researching more independently. With these seminars, where there were interesting topics, it lead to a better turnout versus some which are worded in a way that not many students may understand, hence a lower turnout. Due to this, it would be best when delivering a topic to know who the target audience is so that the title of the topic is understood by that age group/audience.

As for webinars which are events delivered online, these had both advantages and disadvantages. As of COVID-19, seminars were converted into webinars and they were not always as easy to conduct as some speakers spoke faster and went through slides faster without knowing if the audience understood what they were talking about. In addition, it is more difficult to deliver sessions online as many people may not turn their cameras on, thus would lose the factor of face-to-face teaching, and delivery. Furthermore, when the individuals had their cameras off, speakers found it difficult to know if the audience were finding them (the speaker) talking too fast or too slow, whether things need to be repeated, and not knowing if the audience were even present. From the individuals perspective, in relation to the lectures, some users found it difficult to speak up to ask questions.

\subsection{Youtube Videos}











\subsection{Academic Journals}
Mathematics was taught from 6th century BC from the Greeks, however, the topic of communicating mathematics has only been discussed from the sixteenth century. Mathematical communication is understanding how to engage with a range of audiences to learn and understand mathematics. This could be through engagement activities such as an interactive session, exploration of concepts through physical objects, and the standard teaching and answering questions from a textbook.

However, in modern society, many mathematical concepts may not have been interesting or easy to understand, such as algebra, trigonometry in comparison to geometry. The current curriculum within the UK is standardised to help ensure that students can understand the topics with ease through the meticulous planning of what needs to be included within each year group. In contrast, prior to the sixteenth century, mathematical communication was not thought of as scholars were persuaded to explore, understand, and create to compete against other scholars. It would be a battle of who has new ideas and the first to communicate to the society leading to the foundation of mathematics in today’s age.

Mathematical Communication was not spread easily to teach to others as of the lack of a printer. However, in 1440, as seen in [1], this had revolutionised the opportunity to mass produce mathematical research, theorems, and teachings for others to review and understand. Many new scientific ideas were not shared until scholars encountered each other or through connections, hence this meant research was primarily individual and were not seen by others. However, this did not stop individuals from sharing their knowledge, which due to the invention of the printer press, enabled individuals to publish their own scientific research to permanently record it for others to see.

Another study was made to show the relationship of communication for students and their understanding of mathematics. As shown by (Kosko et al. [2]), the author has referenced several resources which showed a positive correlation between the level of a students’ understanding of mathematics and communication in the form of verbal and written. To summarise, for students to understand mathematics, it is vital for them to be able to interact with the concepts so that they can piece the puzzle together. Some may describe and think of mathematics as being logical, however, it requires creativity to be able to think outside of the box and potentially discover new concepts.

A series of methods are detailed within the paper, such as the use of manipulatives (using physical materials), verbal communication and/or social interaction (where students engage with other students to deepen the understanding of mathematical concepts), written communication (enabling them to connect and create a formal understanding of the concept). Consequently, combining both manipulatives and different forms of communication, this builds the solid foundation of the theory. As suggested, without the combination of the two sides, it can decrease the confidence of one understanding the concept fully. Hence, the study primarily shows the importance of manipulatives use and it’s benefits of utilising this to build the platform for students to continuously write and discuss mathematics.

Throughout the study, it compares the different types of communication in mathematics, such as – student discussion, writing about mathematics, and connecting mathematical communication and manipulative use. The blend of these variations in effect allows students to improve and deepen their understanding. The main purpose of manipulatives was used to understand abstract concepts. Within the statistics, there was a positive correlation between manipulative use and writing providing a “statistically significant” result of ρ=.32, p<.01. Discussion was also positive, ρ=.32, p<.01. Finally, the best correlation was between writing and discussion ρ=.32, p<.01. This was shown on page 86, detailed in Table 3. As observed, this can create a feedback loop between the different communication methods.

From the conclusion of the article, it was shown that the use of manipulatives had encouraged students to continue writing about mathematics which further improved several skills other than mathematics, these included social skills, being more logical and creative. However, this study only showed how effective the method given would be, provided to the teachers. The study also showed a range of different methods of teaching, which showed that in a face-to-face setting, students are more engaging and likely to ask more questions. In relation to my project, this would mean that to produce the best results of engagement, and to evaluate the effectiveness, it would be best to create an engagement activity in a face-to-face setting and one online to compare the differences. However, since COVID-19, it may be difficult to conduct the former. On the other hand, this could be conducted within the same bubbles, which can still provide results for the experiment.


From the previous journal [2] the author had referenced (Borasi et al [3]), in which I have included a summary following. As mentioned, [3], the article discussed the “educational value of engaging mathematics students in a specific form of writing to learn – the keeping of a journal throughout a mathematics course.”. It details the positive benefits of keeping a journal which allows students to reflect on the mathematics taught whilst giving teachers an insight on how students feel about the mathematical concepts and the course itself. Similarly, the utilisation of a journal builds a relationship between the teacher and student that is tailored to support each individuals’ learning path.

A general journal has provided many uses of the user confiding in themselves, detailing their feelings and thoughts, as well as acting like a progress log. In this case, if students were open enough to trust their teachers reading this journal, it could provide an insightful view to how each of their students feel about each topic, content, lesson and overall, their progress in relation to the materials covered. However, this is a significant step for students to be able to open up for their teachers to have a look at this journal. Journals are generally private and to create a mathematical journal is a different perspective. Although, if a journal is an extension to the exercise book that they use to complete and answer questions, then this may not be as difficult for students to engage with.

As explained, the education of mathematics was primarily seen to be memorising formulas and equations in order to answer them in exams. However, as students study mathematics to a higher level, it can be seen that this is an inappropriate way of learning as it does not provide understanding but only the application. Furthermore, to fully absorb and engage with mathematics, it is crucial to interact with the content in order to fully understand the meaning of why specific formulas and steps are used. Through memorisation, students may be able to answer specific questions easily as of a similar method used for different types of the same topic. For example, addition with likewise terms (x, y, z, etc). However, if applied the same way for calculus, either differentiation or integration, there are multiple methods in which can solve the statement, more so when using trigonometric functions (cos, sin, tan, arcsin, arctan, etc). Proper understanding allows students to know the why and utilise the methods given to solve the statement efficiently.

If we were taught that Mathematics is a science where there is more solving and logic in contrast to English which uses creativity and writing, it would be seen as surprising to be told to write in maths. However, [3] defends that by “writing mathematics intensively, this can create powerful connections between writing and learning”. The concept of writing to learn is a form of active learning as long as the learner is trying to make meaning of the steps involved for each of their topics. As suggested by (Mayher et al [5]), the concept of active learning is also evidenced by (Oakley 2014, pg, ) [7].

\bibliography
\textbf{Bibliography}
\newline {[1] "Mathematics, Communication, and Community ." Science and Its Times: Understanding the Social Significance of Scientific Discovery. . Encyclopedia.com. 28 Dec. 2021 https://www.encyclopedia.com.}
\newline {[2] Kosko, Karl & Wilkins, Jesse. (2010). Mathematical Communication and Its Relation to the Frequency of Manipulative Use. International Electronic Journal of Mathematics Education. 5. 79-90. 10.29333/iejme/251.}
\newline [3] Borasi, Raffaella, and Barbara J. Rose. “Journal Writing and Mathematics Instruction.” Educational Studies in Mathematics, vol. 20, no. 4, 1989, pp. 347–365., https://doi.org/10.1007/bf00315606
\newline [4] Trisnawati, T., R. Pratiwi, and W. Waziana, 2018: The effect of realistic mathematics education on student's mathematical communication ability. Malikussaleh Journal of Mathematics Learning (MJML), 1, 31, doi:10.29103/mjml.v1i1.741. https://ojs.unimal.ac.id/index.php/mjml/article/view/741
\newline [5] Mayher, J. S., Lester, N. B. and Pradl, G. M.: 1983, Learning to Write/Writing to Learn, Boynton/Cook Publishers, Inc., Upper Montclair, NJ.
\newline [6] Zhou, Jiayu. (2020). A Critical Discussion of Vygotsky and Bruner’s Theory and Their Contribution to Understanding of the Way Students Learn. Review of Educational Theory. 3. 82. 10.30564/ret.v3i4.2444.
\newline [7] Oakley, B., 2014: A mind for numbers. J.P. Tarcher, New York,.





\subsection{Delivering Lectures}
The term lectures is generally used for an educational talk to students who are on further study at university. This could be for undergraduates, postgraduates or doctorates. As of this, the

\section{Teaching platform (Online vs face to face)}

The definition of teaching screen in this instance is the platform. Depending on the location of the teaching, some \textbf{users} (change the word) opt to teach online and face to face, whereas others chose one platform.

\section{Teaching Methods}

The teaching methods are variations of activities that are dependent on the preference of the teacher/presenter as well as the audience for the activity.

For example, for teaching schools, this could be a use of worksheets, whereas masterclasses and engagement days are hands-on interactive activities. Another method would be looking on the screen for talks, etc.

\section{The relationship between the teaching style and method}

\section{Chosen method and style}

After reviewing different teaching styles and methods, the style of communication for mathematics was a presentation. As for communicating data science was a blog.

\section{Adapting to age groups and audiences}

How to make the presentation flexible to suit different age groups and audiences.

\chapter{The National Curriculum}
\section{Overview}
As this project is on the communication of mathematics and data science, the first and foremost important chapter is the national curriculum. As defined [1], The project aim is to create outreach activities that are suitable in engaging with students of appropriate age groups whilst adhering to the standards children should reach in mathematics. However, please note that there is not a fixed chronological order in which it must be taught at schools. As a result, teachers need to be aware that certain topics may not have been covered yet.

\section{Key Stage 3}
Key Stage 3 is the age group of 11 - 14 year olds in Year 7 - 9. My presentation for mathematics is designed in mind to be presented to Year 9 which are 13-14 year olds.

https://www.gov.uk/government/publications/national-curriculum-in-england-mathematics-programmes-of-study/national-curriculum-in-england-mathematics-programmes-of-study#key-stage-3

\section{Key Stage 4}
KS4 is where some children may take GCSEs at the age of 14-15 in Year 10, whereas majority take GCSEs or other national assessments in Year 11 at the age of 15-16.

\section{Key Stage 5}
https://assets.publishing.service.gov.uk/government/uploads/system/uploads/attachment_data/file/516949/GCE_AS_and_A_level_subject_content_for_mathematics_with_appendices.pdf

\chapter{Communicating Mathematics}\label{ch:x.x}

\section{Topic of mathematics}\label{sec:x.x}
The first form of communication was directed at mathematics. Whilst researching, there were a few options that had been considerable when communicating mathematics. Primarily, these were influenced by personal interests and further exploration which led to potentially creating a presentation on 'Music and Maths' and 'Origami Maths'. Consequently, the latter was chosen due to the former being used in communicating data science which with its intrinsic patterns and understanding meant that it felt more in-tune with an older age group - in this instance, Sixth Formers aged 16 - 18. As for communicating mathematics, the topic of 'Origami Maths' felt more hands-on and interactive which led to the lower age group of Year 9s with the teaching of Pythagoras' Theorem.

\section{Preparation of communication}\label{sec:x.x}

%Presentation research

Firstly, the engagement activities will be directed to teach Year 9 onwards. Allowing the ability to see what works best, it can be adapted so that Year 12s are able to learn the topic of mathematics. Year 9 ages are 13-14 in KS3, and Year 12 are 16-17 years old in KS5.

According to the Mathematics programmes of study: key stage 3, page 4, [1], the aim of my project is to use algebra to generalise the structure of arithmetic, including to formulate mathematical relationships. In addition, it would substitute values in expressions, rearrange and simplify expressions, and solve equations and finally use language and properties precisely to analyse numbers, algebraic expressions, 2-D and 3-D shapes, probability and statistics.

Reason mathematically:

extend and formalise their knowledge of ratio and proportion in working with measures and geometry, and in formulating proportional relations algebraically
Geometry and measures

 apply angle facts, triangle congruence, similarity and properties of quadrilaterals to derive results about angles and sides, including Pythagoras’ Theorem, and use known results to obtain simple proofs  use Pythagoras’ Theorem and trigonometric ratios in similar triangles to solve problems involving right-angled triangles
KS4 [2] Geometry and measures

[](https://imgur.com/AZ6tUwR.jpg)
\begin{figure}
  \href{http://commons.wikimedia.org/wiki/File:Pachydyptes_ponderosus.jpg}
       {\includegraphics[width=\textwidth]{picture}}
  \caption{A reconstruction of New Zealand Giant Penguins (\emph{pachydyptes ponderosus}) by
    F.~W.~Kuhnert (1865--1926). Click on the image to visit the page on Wikimedia Commons where it
    was downloaded from.}
\end{figure}

[1] https://assets.publishing.service.gov.uk/government/uploads/system/uploads/attachment_data/file/239058/SECONDARY_national_curriculum_-_Mathematics.pdf

[2] https://assets.publishing.service.gov.uk/government/uploads/system/uploads/attachment_data/file/331882/KS4_maths_PoS_FINAL_170714.pdf

A article was published for the research of Euclidean Constructions and the Geometry of Origami, as shown in [3].

[3] https://www.jstor.org/stable/2690924?seq=1#metadata_info_tab_contents

As mentioned in [4], there are many resources in teaching students mathematics using origami. [4] has mentioned a variety of topics that can be covered and provides links for inspiration when creating their own material.

[4] http://www.paperfolding.com/math/

Trisecting an angle was one thing that could have been shown [5], but as I was looking to understand and present the topic of trigonometry, this has been forgone.

[5] https://plus.maths.org/content/trisecting-angle-origami

This [6] was one of the initial pages which had provided me the links to [4].

[6] https://riverbendmath.org/modules/Origami/Sonobe_Polyhedra/Links/teaching

This was [7] providing a collection of almost 200 single concept lessons. It was interesting to research and review them as it gave an insight to how to teach students.

[7] https://mypages.iit.edu/~smile/mathinde.html

I found a resource [8] asking about the “Math in Origami”, detailing a few different things that we could teach.

[8] http://sigmaa.maa.org/mcst/documents/ORIGAMI.pdf

[9] was a brilliant resource as it detailed many links as well. This helped to ensure that I could figure out what I would like to include.

[9] https://www.artfulmaths.com/origami-in-lessons.html

This was also similar to [5], as from the same website, it showed the trisect of an angle. [10] https://plus.maths.org/content/power-origami

\section{Making a presentation}\label{sec:x.x}

Making the presentation slides

\section{Realisation of other things to prepare}\label{sec:x.x}

Preparing a detailed handover, lesson plans, handout sheets, presentation slides

\section{Practice communication}\label{sec:x.x}

Practicing the presentation beforehand

Leading up to the the presentation that was to be presented to the origami society, I had been practicing beforehand with a friend. During this process, I was consistently making changes whilst understanding what needs to be solidified before the presentation at the origami society. I was moving a lot to practice speaking. It was more overwhelming than I had thought as I wanted to try to make it perfect. My energy was seen but my knowledge content was still lacking. Furthermore, I was rambling during the start of the introduction which meant I was taking a lot longer for the slides that were not important. In this case, I was waffling and unable to demonstrate as well as I intended to. My face was making a lot of emotions but it was not always ideal for a school situations.

\section{Reflections}\label{sec:x.x}

\chapter{Communicating Data Science}\label{ch:x}

%The text goes here ...

\section{Literature Review}
From [1], it can be seen that it is important to use headings so that the audience can see step-by-step what the procedure is when explaining concepts. There are bold headings and subheadings so that it is easier for the reader to find the relevant information. It has a conversational language tone to engage with the reader.

As seen in [2], there is also an introduction to the topics covered in the blog. It includes the code snippets in a code box to make it easier for the readers to test out the code themselves by being able to copy the code easily. It is more interactive for some of the images which shows how it is done. For example, there is an animated image to show where the cursor moves, the graph changes. Each blog summaries and provides further reading for the reader to explore if they were interested.

On the website [3], which is the main page for most of the references used, it has shown the range of ways to start the blog. Some have included the heading “Introduction” [4], whereas others [5] and [6] removes the heading and still includes the introduction. Another way of starting a blog was seen in [7], where they started it really short with “Recently, I watched this Ted Talk”, and proceeds to embed the video in the blog. This creates a different perspective to other blogs as it encourages the reader to potentially watch the video, or if not, they’ve explained it below the video itself.

Important things to note/remember are written in bold as shown in [8] and ends with a recap of what the reader has learnt from the following blog. In [9], it also shows certain words in bold which capture the importance of the concepts learnt.

\bibliography
\textbf{Bibliography}
\newline [1] https://towardsdatascience.com/probability-concepts-explained-maximum-likelihood-estimation-c7b4342fdbb1
\newline [2] https://towardsdatascience.com/interpretml-another-way-to-explain-your-model-b7faf0a384f8
\newline [3] https://towardsdatascience.com/
\newline [4] https://towardsdatascience.com/collect-data-from-twitter-a-step-by-step-implementation-using-tweepy-7526fff2cb31?source=collection_home---------5-------------------------------
\newline [5] https://towardsdatascience.com/opening-the-black-box-an-explanation-of-explainable-ai-7024d8b1f6b3?source=collection_home---------6-------------------------------
\newline [6] https://towardsdatascience.com/cultivating-a-technical-mindset-for-data-analysts-c19eb6090089?source=collection_home---------7-------------------------------
\newline [7] https://towardsdatascience.com/towards-greater-purpose-in-ai-research-91218cd108ff?source=collection_home---------8-------------------------------
\newline [8] https://towardsdatascience.com/fasterai-a-library-to-make-smaller-and-faster-neural-networks-70c3ff2e2ba3
\newline [9] https://towardsdatascience.com/real-time-speech-recognition-python-assemblyai-13d35eeed226

\section{Form of communication}\label{sec:x.x}

Communicating mathematics to sixth form students in the form of a blog post.

\section{Understand how to present a blog post}\label{sec:x.x}

%... goes here.

\section{Preparation of a blog post}\label{sec:x.x}


\section{Making the blog post}\label{sec:x.x}


\chapter{Final Remarks}\label{ch:concl}
\section{Project Findings}
Talk about maths and data science

\section{Conclusions}
Conclusion restates the main arguments, tells about the consequences, and provides suggestions for future work.

\section{Project Reflections}
Talk about what you learnt and how you felt about the project

\chapter*{Appendix}
\appendix
\addcontentsline{toc}{chapter}{Appendix}

\section*{Detailed Handover}
Insert detailed table of lesson plan next to presentation slides

\section*{Lesson Plan}
Insert lesson plan

\section*{Handout sheets}\label{secx}
Insert handouts

\section*{Presentation}\label{secx.x}
Insert presentation slides

\section*{Presentation video}\label{secx.x}
Insert video link of presentation

%%% Those two bibliography items are given as an example remove them and add your bibliography items below  %%%

\begin{thebibliography}{999}

\bibitem{Noether}
E.~Noether.
\newblock Invariante {V}ariationsprobleme.
\newblock {\em Nachr. d. K{\"o}nig. Gesellsch. d. Wiss. zu G{\"o}ttingen,
  Math-phys. Klasse, {Seite 235-157}}, 1918.

\bibitem{Turing}
A.~M. Turing.
\newblock Computing machinery and intelligence.
\newblock {\em Mind}, 59:433--460, 1950.

% check if i can reference either in sections

\end{thebibliography}

\section*{Acknowledgements}
I would like to thank my supervisor Dr. Joe Bailey for his guidance and support during the period of writing my dissertation, my personal tutors Dr. Jess Claridge and Professor Peter Higgins for supporting me throughout the course of the degree and their advice to help me get to where I am today and finishing my modules and this project in time.

\end{document}
