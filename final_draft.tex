\documentclass[12pt, a4paper,oneside]{book}
\newcommand{\re}{\mathrm{e}}
\newcommand{\ri}{\mathrm{i}}
\newcommand{\rd}{\mathrm{d}}
\def\semicolon{\nobreak\mskip2mu\mathpunct{}\nonscript\mkern-\thinmuskip{;}
\mskip6muplus1mu\relax} % This defines the semicolon command

        % allows index generation
\usepackage{graphicx}        % standard LaTeX graphics tool
                             % when including figure files
\usepackage{multicol}        % used for the two-column index




\usepackage{color,tikz}
%\usepackage[unicode,bookmarks,bookmarksopen,bookmarksopenlevel=2,colorlinks,linkcolor=blue,citecolor=green]{hyperref}

\usepackage{amsmath,eucal,amssymb}
\usepackage{mathrsfs,graphicx,texdraw}
\usepackage{fancyhdr,framed}
\usepackage{tikz, tikz-3dplot}
\usepackage{tkz-euclide}
\usetikzlibrary{decorations.fractals}
\usetikzlibrary{decorations.footprints}

\usepackage{palatino}

\usepackage[latin1]{inputenc}

\usepackage[T1]{fontenc}
%\usepackage[dvips]{graphicx}
%\usepackage{times}


\definecolor{mygreen}{HTML}{622567} %%% Purple
\definecolor{Gray}{HTML}{333333} %%% Gray

\definecolor{myred}{HTML}{D55C19} %%%EssexOrange
\definecolor{myblue}{HTML}{007A87} %%%Seagrass
\definecolor{Mint}{HTML}{35C4B5} %%% Mint


\def\grole{\mathrel{\mathchoice {\vcenter{\offinterlineskip\halign{\hfil
$\displaystyle##$\hfil\cr>\cr\noalign{\vskip-1.5pt}<\cr}}}
{\vcenter{\offinterlineskip\halign{\hfil$\textstyle##$\hfil\cr
>\cr\noalign{\vskip-1.5pt}<\cr}}}
{\vcenter{\offinterlineskip\halign{\hfil$\scriptstyle##$\hfil\cr
>\cr\noalign{\vskip-1pt}<\cr}}}
{\vcenter{\offinterlineskip\halign{\hfil$\scriptscriptstyle##$\hfil\cr
>\cr\noalign{\vskip-0.5pt}<\cr}}}}}

\newenvironment{rcases}
  {\left.\begin{aligned}}
  {\end{aligned}\right\rbrace}

\newenvironment{lcases}
  {\left\lbrace\begin{aligned}}
  {\end{aligned}\right.}

\newtheorem{theorem}{Theorem}[section]
\newtheorem{lemma}[theorem]{Lemma}
\newtheorem{proposition}[theorem]{Proposition}
\newtheorem{corollary}[theorem]{Corollary}
\newtheorem{definition}[theorem]{Definition}
\newtheorem{example}[theorem]{Example}
\newtheorem{remark}[theorem]{Remark}

\newenvironment{proof}[1][Proof]{\begin{trivlist}
\item[\hskip \labelsep {\bfseries #1}]}{\end{trivlist}}
\newenvironment{solution}[1][Solution]{\begin{trivlist}
\item[\hskip \labelsep {\bfseries #1}]}{\end{trivlist}}

\newcommand{\qed}{\nobreak \ifvmode \relax \else
      \ifdim\lastskip<1.5em \hskip-\lastskip
      \hskip1.5em plus0em minus0.5em \fi \nobreak
      \vrule height0.75em width0.5em depth0.25em\fi}

\numberwithin{equation}{section}
\def\Ad{{\mbox{Ad}}}
\def\im{{\mbox{Im}}}
\def\Re{{\mbox{Re}\;}}
\def\ad{\mathrm{ad\,}}
\def\openone{\leavevmode\hbox{\small1\kern-3.3pt\normalsize1}}
\def\Res{\mathop{\mbox{Res}\,}\limits}
\def\biglb{\big[\hspace*{-.7mm}\big[}
\def\bigrb{\big]\hspace*{-.7mm}\big]}


\def\bigrbt{\mathop{\bigrb }\limits_{\widetilde{\;}}}
\def\biggrbt{\mathop{\biggrb }\limits_{\widetilde{\;}}}
\def\Bigrbt{\mathop{\Bigrb }\limits_{\widetilde{\;}}}
\def\Biggrbt{\mathop{\Biggrb }\limits_{\widetilde{\;}}}

\def\bPhi{\mathbf{\Phi}}
\def\bM{\mathbf{M}}
\def\bm{\mathbf{m}}
\def\bbbc{{\Bbb C}}
\def\bbbr{{\Bbb R}}
\def\bbbz{{\Bbb Z}}
\def\bbbs{{\Bbb S}}
\def\diag{\mbox{diag}\,}
\def\tr{\mbox{tr}\,}

\textwidth=17cm   \textheight=24.5cm \voffset=-2cm

\evensidemargin=-0.5cm \oddsidemargin=-0.5cm

 \renewcommand{\baselinestretch}{1.5}

%%%%%%%%fancy header%%%%%%%%%%%%%%%%%

\usepackage{fancyhdr}
\pagestyle{fancy}
\usepackage{calc}
\newlength{\pageoffset}
\setlength{\pageoffset}{0cm}% use whatever you like
\fancyheadoffset[LE,RO]{\pageoffset}
\renewcommand{\chaptermark}[1]{\markboth{#1}{}}
\renewcommand{\sectionmark}[1]{\markright{\thesection\ #1}}
\fancyhf{}
\fancyhead[LE]{\makebox[\pageoffset][l]{\thepage}\hfill\leftmark}
\fancyhead[RO]{\rightmark\hfill\makebox[\pageoffset][r]{\thepage}}
\fancypagestyle{plain}{%
    \fancyhead{} % get rid of headers
    \renewcommand{\headrulewidth}{0pt} % and the line
}

%%%%%%%%%%%%%%%%%%%%%%%%%%%%%%%%%%%%

\arraycolsep=2pt

%%%%Fancy Chapter%%%

%\usepackage{lmodern}
\usepackage{graphicx}
\usepackage{xcolor}
\usepackage{titlesec}
\usepackage{microtype}
\usepackage{lipsum}

%\definecolor{myblue}{RGB}{0,82,155}

\titleformat{\chapter}[display]
  {\normalfont\bfseries\color{myred}}
  {\filleft\hspace*{-60pt}%
    \rotatebox[origin=c]{90}{%
      \normalfont\color{black}\Large%
        \textls[180]{\textsc{\chaptertitlename}}%
    }\hspace{10pt}%
    {\setlength\fboxsep{0pt}%
    \colorbox{myred}{\parbox[c][3cm][c]{2.5cm}{%
      \centering\color{white}\fontsize{80}{90}\selectfont\thechapter}%
    }}%
  }
  {10pt}
  {\titlerule[2.5pt]\vskip3pt\titlerule\vskip4pt\LARGE\sffamily}

% hyperref should be loaded after all other packages
% aliascnt is needed to get \autoref (from hyperref) to work correctly with custom amsthm theorems
\usepackage{aliascnt}
\usepackage[colorlinks,linkcolor=myblue,citecolor=mygreen]{hyperref}


%%% Make Necessary Changes Here Only %%%

\newcommand{\MyName}{Your Name} %% Change this to your name %%
\newcommand{\CapstoneVariant}{MAXXX} %% Change this to the capstone project variant you enrolled %%
\newcommand{\ProjectTitle}{Title of Your Project} %% Change this to your project title %%
\newcommand{\SuperviserName}{Your Supervisor}  %% Change this to your project supervisor %%

%%%%%%%%%%%%%%%%%%%%%%%%%%

%%% Do not change anything here %%%
\begin{document}

\thispagestyle{empty}

\begin{minipage}{0.2\textwidth}
\centerline{\includegraphics[width=.4\textwidth]{essex} }
\end{minipage}
\begin{minipage}{0.8\textwidth}

$ \qquad \qquad \qquad ${\LARGE \bf \sl University of Essex}

{\LARGE \bf Department of Mathematical Sciences}

\end{minipage}

\begin{center}

\noindent\textcolor{myred}{\rule{\linewidth}{4.8pt}}

\vspace{1cm}

{\LARGE \sc MA838: Capstone  Project Dissertation}

\vspace{1.5cm}

{\Huge{\color{myblue} Communicating Mathematics \\ and Data Science}}

\vspace{1.5cm}


{\Large \bf Karen Yim \\ 1705960}

\vspace{1.5cm}

%\centerline{\includegraphics[width=1\textwidth]{Figure}}
\vspace{3cm}

\vspace{1.5cm}

{\Large {Supervisor:} {\color{mygreen} \bf  Dr Joe Bailey}}

\vspace{.25cm}

\noindent\textcolor{myred}{\rule{\linewidth}{4.8pt}}

\vspace{1cm}
{\Large \today }\\[4pt]
{\Large Colchester}

\end{center}

\newpage

\section*{Abstract}
%Abstract of approximately 300 words summarizes the aims, scope, and the conclusions of a dissertation.

%Abstract (500 words roughly), brief rationale (meaning behind) why are you doing the study, method (explain audience, age, whatever, like experiment), results (data findings), discussions (conclusions, explain what the findings mean in context, how is it applied to the research project), limitations (what would you do to improve, or further research)

The aim of the project is to conduct a study on how to engage with students on communicating mathematics and data science. I've chosen this project as of previous interest in mathematics education and the possibility of showcasing new career options and topics to share to different age groups. The main project is communicating mathematics and data science which will be divided into sections as to use two different engagement methods. As a result

\tableofcontents
%\listoffigures
%\listoftables

%%% Do not change anything  above %%%
\chapter{Introduction}\label{ch:1}
%% You can start with some introduction of your project and background.
My project consists of two sections, the first is communicating mathematics to Year 9 students on the topics of Origami and Mathematics. The second is to communicate data science to Sixth Formers on the {\color{red}\textbf{topic of data science with the Spotify API.}} (need to reword) {\color{blue}\textbf{note: this might be better suited to another section such as abstract perhaps.}}

To communicate mathematics and data science is to ask - how did I find mathematics in secondary school and how was it interesting to learn? Many have spoken about the difficulties of understanding mathematics as it can be too "logical" that the more creative students were unable to comprehend. However, I believe this to be {\color{red}\textbf{wrong}} if the teacher was able to evidently show the similarities between the arts and sciences. As of this, I have chosen to put my dedication into mathematics education and specifically on combining and sharing to others the benefits and/of how similar arts and sciences are. In this case, it will be origami and mathematics to communicate mathematics and using the Spotify API to communicate data science.

{\color{blue}\textbf{Maybe put the bottom part of introduction into a new section known as Background?}}

Firstly, {\color{blue}\textbf{(perhaps remove "Firstly", not sure yet)}} prior to choosing the project, I was previously interested in learning new things to expand my horizon such as different maths topics. Hence, I spent a number of hours watching numerous YouTube videos in the background or sometimes more focused in the topic I was intrigued by. When the project briefs were shown, I had an interest in the project by Professor Abdel Salhi - "Error modelling in Origami construction". Consequently, I wanted to learn more about what origami maths was and proceeded to conducting a research on the topic.

Unfortunately, I did not choose the project in the end as I realised there was another project that could enable me to show the creative side of maths through the project "Communicating Mathematics/Data Science" by Dr. Joe Bailey. This latter project would subsequently let me explore the different engagement methods of communication as well as showcase my knowledge and interest of creativity bonding with mathematics. {\color{red}\textbf{(This latter project would subsequently hone my knowledge and connect with my musical side to mathematics.)}}

Prior to applying for university, I was captivated by music and maths, which the composer John Cage had influenced me to this joint degree. His compositions were inspired by one number, or a set of numbers which would result in his piece showcasing that specific number. My interest for this topic further expanded when I came across the book "Music of the Spheres".

This book showed that the arts and sciences were undeniably closely related and similar before they diverged into their respective areas in this modern society. Those who are from either of the fields may have discrepancies, however they would not be able to work without the other. This {\color{red}\textbf{meant}} (change the word) that to make the findings of science and logic required creativity and vice versa. In this modern society, many may believe this to be preposterous, it would be outrageous that they are linked. This project aims to {\color{red}\textbf{affirm}} (change the word) that arts and sciences go hand-in-hand with each other.

\newline {\color{blue}\textbf{another version -------------------------------}}
\newline To write about the project is to first introduce what the project is. This project is on communicating mathematics and data science with a general overview and a focus on using two teaching methods to different audiences as a comparison between different forms of teaching styles. This dissertation will expand on the process of understanding effective and engaging pedagogy in mathematics, different types of teaching styles and methods, preparation and conclusions of chosen pedagogy styles, and the use of mathematical language and how to interact with different audiences.

To communicate mathematics and data science is to ask - how did I find mathematics in secondary school and how was it interesting to learn? Many have spoken about the difficulties of understanding mathematics as it can be too "logical" that the more creative students were unable to comprehend. However, I believe this to be {\color{red}wrong} if the teacher was able to evidently show the similarities between the arts and sciences. As of this, I have chosen to put my dedication into mathematics education and specifically on combining and sharing to others the benefits and/of how similar arts and sciences are. In this case, it will be origami and mathematics to communicate mathematics and using the Spotify API to communicate data science.

As of this, to first understand on how to communicate the two subjects, I will conduct a research on pedagogy for mathematics. Then, to communicate mathematics, I've chosen to create a presentation as to engage with Year 9 students (13-14 year olds) with handouts and interactive activities to learn about Pythagoras' Theorem using origami. Next, I will communicate data science in the form of a blog post detailing the specifics for a Spotify API (Application Programming Interface) aimed towards Sixth Formers (16-18 year olds). Finally, the last chapter entails the project findings, conclusions, and project reflections.

\chapter{Literature Review/Pedagogy Research}\label{ch:2}

Chapter 1 will detail the background research carried out in preparation for the project of communicating mathematics and data science. In order to teach mathematics, one must undertake preparation to fully understand {\color{red}\textbf{the aims of the project and how to proceed.}} (how to teach and what is required before teaching a class.)  As a result, the first preparation was a series of reviewing different academic journal, books, online media and personal experience on teaching maths.

\newline {\color{blue}\textbf{another version -------------------------------}}

When conducting the literature review, I started it off with what I had known about teaching in mathematics in different forms of media, however I had not known what pedagogy meant. Hence, the first step was to understand the word "Pedagogy".
The definition of pedagogy (from Oxford Language Dictionary) is "The method and practice of teaching, especially as an academic subject or theoretical concept.".

Prior to the project and researching pedagogy, I was always interested in learning new topics and expanding my knowledge of(or on?) different maths topics. As of this, I was exposed to various speakers who presented numerous topics on mathematics. Taken from previous {\color{red}\textbf{exposure}}, I was revisiting and recollecting my experiences when attending these talks - either watching online, live online, or face-to-face. This had provided significant support to my project and I would continue to spend the rest of the time when conducting the research on other platforms for the project. To expand on the experiences, it includes masterclasses, open day talks, induction talks, summer schools, and live videos.

To explain the different sections, it will be following the age group that I was in whilst attending those events. In this case, the first one would be the ones I attended whilst at secondary school aged 11-16. Firstly, the ones from when I was in secondary school, so I would be between the years of Year 9/10/11, or it might be in the first year of sixth form at age 16/17, where I remember going to some masterclasses at Greenwich University for mathematics masterclasses. This was interesting because we would have had some knowledge of A-level maths and was learning more about that. This would be an engagement activity and had been explained with a depth that A-level students would understand but with still a bit of difficulty in the concepts that are for university students to study. Whilst it was difficult to understand everything, it was firstly explained in a more basic manner and continued to be a fun topic so that students would be engaged and interactive during the masterclass. It involved doing activities in between explanations and was for the day with breaks and tours.



\section{Teaching Styles}\label{sec:2.1}

To understand the different teaching styles, I had been researching primarily on the resource of YouTube as well as from personal experience. As a result, it can be detailed that there are lectures, lessons, engagement activities, masterclasses, talks, summer schools, and workshops.

\subsection{Delivering Lectures}
The term lectures is generally used for an educational talk to students who are on further study at university. This could be for undergraduates, postgraduates or doctorates. As of this, the

\subsection{Teaching platform (Online vs face to face)}

The definition of teaching screen in this instance is the platform. Depending on the location of the teaching, some \textbf{users} (change the word) opt to teach online and face to face, whereas others chose one platform.

\subsection{Teaching Methods}

The teaching methods are variations of activities that are dependent on the preference of the teacher/presenter as well as the audience for the activity.

For example, for teaching schools, this could be a use of worksheets, whereas masterclasses and engagement days are hands-on interactive activities. Another method would be looking on the screen for talks, etc.

\section{The relationship between the teaching style and method}

\section{Chosen method and style}

After reviewing different teaching styles and methods, the style of communication for mathematics was a presentation. As for communicating data science was a blog.

\section{Adapting to age groups and audiences}

How to make the presentation flexible to suit different age groups and audiences.

\chapter{The National Curriculum}
\section{Overview}
As this project is on the communication of mathematics and data science, the first and foremost important chapter is the national curriculum. As defined [1], The project aim is to create outreach activities that are suitable in engaging with students of appropriate age groups whilst adhering to the standards children should reach in mathematics. However, please note that there is not a fixed chronological order in which it must be taught at schools. As a result, teachers need to be aware that certain topics may not have been covered yet.

\section{Key Stage 3}
Key Stage 3 is the age group of 11 - 14 year olds in Year 7 - 9. My presentation for mathematics is designed in mind to be presented to Year 9 which are 13-14 year olds.

https://www.gov.uk/government/publications/national-curriculum-in-england-mathematics-programmes-of-study/national-curriculum-in-england-mathematics-programmes-of-study#key-stage-3

\section{Key Stage 4}
KS4 is where some children may take GCSEs at the age of 14-15 in Year 10, whereas majority take GCSEs or other national assessments in Year 11 at the age of 15-16.

\section{Key Stage 5}
https://assets.publishing.service.gov.uk/government/uploads/system/uploads/attachment_data/file/516949/GCE_AS_and_A_level_subject_content_for_mathematics_with_appendices.pdf

\chapter{Communicating Mathematics}\label{ch:x.x}

\section{Topic of mathematics}\label{sec:x.x}
The first form of communication was directed at mathematics. Whilst researching, there were a few options that had been considerable when communicating mathematics. Primarily, these were influenced by personal interests and further exploration which led to potentially creating a presentation on 'Music and Maths' and 'Origami Maths'. Consequently, the latter was chosen due to the former being used in communicating data science which with its intrinsic patterns and understanding meant that it felt more in-tune with an older age group - in this instance, Sixth Formers aged 16 - 18. As for communicating mathematics, the topic of 'Origami Maths' felt more hands-on and interactive which led to the lower age group of Year 9s with the teaching of Pythagoras' Theorem.

\section{Preparation of communication}\label{sec:x.x}

Presentation research

\section{Making a presentation}\label{sec:x.x}

Making the presentation slides

\section{Realisation of other things to prepare}\label{sec:x.x}

Preparing a detailed handover, lesson plans, handout sheets, presentation slides

\section{Practice communication}\label{sec:x.x}

Practicing the presentation beforehand

Leading up to the the presentation that was to be presented to the origami society, I had been practicing beforehand with a friend. During this process, I was consistently making changes whilst understanding what needs to be solidified before the presentation at the origami society. I was moving a lot to practice speaking. It was more overwhelming than I had thought as I wanted to try to make it perfect. My energy was seen but my knowledge content was still lacking. Furthermore, I was rambling during the start of the introduction which meant I was taking a lot longer for the slides that were not important. In this case, I was waffling and unable to demonstrate as well as I intended to. My face was making a lot of emotions but it was not always ideal for a school situations.

\section{Reflections}\label{sec:x.x}

\chapter{Communicating Data Science}\label{ch:x}

%The text goes here ...

\section{Form of communication}\label{sec:x.x}

Communicating mathematics to sixth form students in the form of a blog post.

\section{Understand how to present a blog post}\label{sec:x.x}

%... goes here.

\section{Preparation of a blog post}\label{sec:x.x}


\section{Making the blog post}\label{sec:x.x}


\chapter{Final Remarks}\label{ch:concl}
\section{Project Findings}
Talk about maths and data science

\section{Conclusions}
Conclusion restates the main arguments, tells about the consequences, and provides suggestions for future work.

\section{Project Reflections}
Talk about what you learnt and how you felt about the project

\chapter*{Appendix}
\appendix
\addcontentsline{toc}{chapter}{Appendix}

\section*{Detailed Handover}
Insert detailed table of lesson plan next to presentation slides

\section*{Lesson Plan}
Insert lesson plan

\section*{Handout sheets}\label{secx}
Insert handouts

\section*{Presentation}\label{secx.x}
Insert presentation slides

\section*{Presentation video}\label{secx.x}
Insert video link of presentation

%%% Those two bibliography items are given as an example remove them and add your bibliography items below  %%%

\begin{thebibliography}{999}

\bibitem{Noether}
E.~Noether.
\newblock Invariante {V}ariationsprobleme.
\newblock {\em Nachr. d. K{\"o}nig. Gesellsch. d. Wiss. zu G{\"o}ttingen,
  Math-phys. Klasse, {Seite 235-157}}, 1918.

\bibitem{Turing}
A.~M. Turing.
\newblock Computing machinery and intelligence.
\newblock {\em Mind}, 59:433--460, 1950.

\end{thebibliography}

\section*{Acknowledgements}
I would like to thank my supervisor Dr. Joe Bailey for his guidance and support during the period of writing my dissertation, my personal tutors Dr. Jess Claridge and Professor Peter Higgins for supporting me throughout the course of the degree and their advice to help me get to where I am today and finishing my modules and this project in time.

\end{document}
