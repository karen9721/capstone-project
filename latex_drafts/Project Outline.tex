\documentclass{article}
\usepackage[utf8]{inputenc}

\setlength{\parskip}{1em}

\title{ Project Outline \\{\Large MA838-FY}\\{\Large Professor Joseph Bailey} }
\author{Karen Yim \\ 1705960 }
\date{October 2021}

\begin{document}

\maketitle

\section{Introduction}
The Capstone Project chosen is known as Communicating Mathematics / Data Science which consists of 45 credits lasting over the full year. The aim of this project is to engage with students in Key Stages 3 and 4 or above on the subjects of Mathematics and Data Science through a series of activities to demonstrate extensive knowledge of pedagogy and mathematical / data science content.

\section{Stages of the Project}
To summarise, there will be four stages in this project - the research of pedagogy, decision of mathematics and data science topics, preparation and engagement activities, and lastly, the dissertation on the project itself. Pedagogy will be the main focus for autumn term to delve into the various methods of teaching and engagement with students as a goal of making mathematics interesting and fun. For the purpose of researching pedagogy, there will be discussions of engagement talks and activities, the National Curriculum, academic journals, and other subsequent materials.

Next, there will be two types of different engagement activities to share knowledge on specific topics of Mathematics and Data Science to Secondary School students between the year groups of Year 9 and 10 for Mathematics and potentially Year 11 above for Data Science. The different ways of engagement can bring a comparison of methods and detail the importance of pedagogy.

For Mathematics, the aim is to visualise to students mathematics in another light through origami maths. Within mathematics, this will be two sections - pedagogy in the Autumn term and presentation in the Spring term. The goal is to show secondary school or sixth form students the mathematics of origami, creating a visual and hands-on presentation. The presentation will be adaptable to the audience and tailored to understand complex mathematical concepts without difficulties for the target audience whilst being interactive for them to explore on their own.

For Data Science, it will be a blog post or another type of engagement activity to demonstrate the understanding of data science to students. This will involve the analysis of a music Application Programming Interface (API), explaining the code, classification methods, and techniques used. Albeit it is linked between the two subjects, it will be treated as separate projects in order for ease of explanation and showing the different types of activities possible through the comparison at the end of the dissertation.

The dissertation will detail how the process of the project has been, providing an account of what went well, what could be improved and the findings from the project. Current progress will be updated with the supervisor weekly to keep on track and ensure that the relevant information has been been documented, providing feedback when revising targets and goals throughout the terms.

It is to note that prior to the start of the project, there has been interest in origami mathematics, as well as the connection between music, mathematics and data science. Due to this, the project was chosen to further explore these pathways to compare a range of topics and methods whilst engaging with an interactive audience.

\end{document}
