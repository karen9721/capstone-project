\documentclass{article}
\usepackage[utf8]{inputenc}
\usepackage{color}

\title{MA838 no organising, just writing}
\author{Karen Yim}
\date{April 2022}

\begin{document}

\maketitle

\section{Introduction}
To write about the project is to first introduce what the project is. This project is on communicating mathematics and data science with a general overview and a focus on using two teaching methods to different audiences as a comparison between different forms of teaching styles. This dissertation will expand on the process of understanding effective and engaging pedagogy in mathematics, different types of teaching styles and methods, preparation and conclusions of chosen pedagogy styles, and the use of mathematical language and how to interact with different audiences.

To communicate mathematics and data science is to ask - how did I find mathematics in secondary school and how was it interesting to learn? Many have spoken about the difficulties of understanding mathematics as it can be too "logical" that the more creative students were unable to comprehend. However, I believe this to be {\color{red}wrong} if the teacher was able to evidently show the similarities between the arts and sciences. As of this, I have chosen to put my dedication into mathematics education and specifically on combining and sharing to others the benefits and/of how similar arts and sciences are. In this case, it will be origami and mathematics to communicate mathematics and using the Spotify API to communicate data science.

As of this, to first understand on how to communicate the two subjects, I will conduct a research on pedagogy for mathematics. Then, to communicate mathematics, I've chosen to create a presentation as to engage with Year 9 students (13-14 year olds) with handouts and interactive activities to learn about Pythagoras' Theorem using origami. Next, I will communicate data science in the form of a blog post detailing the specifics for a Spotify API (Application Programming Interface) aimed towards Sixth Formers (16-18 year olds). Finally, the last chapter entails the project findings, conclusions, and project reflections.

\section{Pedagogy Research}
When conducting the literature review, I started it off with what I had known about teaching in mathematics in different forms of media, however I had not known what pedagogy meant. Hence, the first step was to understand the word "Pedagogy".
The definition of pedagogy (from Oxford Language Dictionary) is "The method and practice of teaching, especially as an academic subject or theoretical concept.".

Prior to the project and researching pedagogy, I was always interested in learning new topics and expanding my knowledge of(or on?) different maths topics. As of this, I was exposed to various speakers who presented numerous topics on mathematics. Taken from previous exposure, I was revisiting and recollecting my experiences when attending these talks - either watching online, live online, or face-to-face. This had provided significant support to my project and I would continue to spend the rest of the time when conducting the research on other platforms for the project. To expand on the experiences, it includes masterclasses, open day talks, induction talks, summer schools, and live videos.

To explain the different sections, it will be following the age group that I was in whilst attending those events. In this case, the first one would be the ones I attended whilst at secondary school aged 11-16. During this period, we were able to attend

Projects of this nature would require students to thoroughly understand a specific problem
or area of mathematics, data science or computing, including the background/history of the
problem, solutions, where the problem fits in the wider mathematical world. A major aim
would be to produce materials (either through an interactive activity, class session, engaging
talk, video resource) that can demonstrate the importance of the problem, the reason why it
is interesting/important but also explain the solution in a clear and concise manner whilst
still being interesting and engaging.

The audience would most likely be to secondary school
students (Yr 9 to Y13), however the materials should be able to be tweaked to be presented
to general lay audiences (students interested in working with primary school children can
also apply).

As the project will involve eventually demonstrating and presenting (in some
form) the activity produced there will be opportunities to join in the Departments outreach
offering by taking part in school visits/public talks with the

initial aim of understanding
what makes a good talk and how activities can be put together in an engaging manner and
importantly how to interact with students/members of the public.

\end{document}
